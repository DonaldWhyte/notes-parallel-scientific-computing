%
% Name: Parallel and Scientific Computing Module Book (2013-2014)
% Author: Donald Whyte (sc10dw@leeds.ac.uk)
%

\documentclass{article}

\usepackage[margin=2cm]{geometry} % easy page formatting
	\geometry{letterpaper}
\usepackage{doc} %special logo commands
\usepackage{url} % formatting URLs
\usepackage{datetime} % up-to-date, automatically generated times
% For graphic files
\usepackage{graphicx}
\usepackage{epstopdf}
\DeclareGraphicsRule{.tif}{png}{.png}{`convert #1 `dirname #1`/`basename #1 .tif`.png}

% Set the title, author, and date.
\title{Parallel and Scientific Computing \\ COMP3920}
\author{Donald Whyte}
\date{\today}

% The document proper.
\begin{document}

% Add the title section.
\maketitle

% Add various lists on new pages.
\pagebreak
\tableofcontents

\pagebreak
\listoffigures

\pagebreak
\listoftables

% Start the paper on a new page.
\pagebreak

\section{Scientific Computing}

Scientific Computing is the study of methods and procedures to approximate solutions to mathematical problems.Often, we \textit{have} to approximate solutions because computing the exact/optimal solution takes too long or is not possible to due machine inaccuracy. This module deals with how to use \textbf{mathematical models} to derive \textbf{numerical models} that can be then be computed/solved in parallel, across multiple machines. There are three key issues to deal with when trying to solve scientific problems -- accuracy, efficiency and reliability.

\textbf{Accuracy} refers to the accuracy of the final solution and can appear in many guises:
\begin{enumerate}
	\item \textbf{Modelling Error} -- occurs when the mathematical model representing the real world problem is an approximation/simplification (e.g. assumption that the Earth is  sphere)
	\item \textbf{Discretisation Error} -- occurs when the problem is approximated further in order to solve it numerically on a computer (cannot have true real numbers  due to memory limitations and floating point imprecision).
	\item \textbf{Imprecise Data} -- external information being used in the simulation is incorrect, incomplete or an approximation
	\item \textbf{Inexact Arithmetic} -- computation has increasing error due to floating point arithmetic
	\item \textbf{Human Error} -- the human made an error in one of the stages above
\end{enumerate}

\textbf{Efficiency} refers to number of resources required to solve the problem, whether those resources are time, memory, machines and so on. This should be be optimised in terms of the:
\begin{enumerate}
	\item number of arithmetic operations used, as well as how they grow with the problem size and how these operations can be efficiently distributed across multiple processes)
	\item memory used
	\item running time (time taken for the program toe execution)
\end{enumerate}

\textbf{Reliability} is also an important factor. Ideally, a given numerical method could be applied in such a way that it guarantees:
\begin{enumerate}
	\item a given level of accuracy for \textit{any} computation
	\item \textbf{robustness}. That is, the success/accuracy of the method does not dependant heavily on initial data and control parameters (e.g. initial iterate in Jacobi iteration
	\item \textbf{convergence to a solution}. That is, it always works (or at least fails gracefully instead of running forever).
	
\end{enumerate}

TODO: Moore's law and sustaining growth

\subsection{Finite Precision Arithmetic}

TODO

\subsection{Linear Algebra}

\subsubsection{Matrix Addition}

TODO: algorithms, complexity and how to parallelise

TODO: iso-efficiency and $t_{comp}$/$t_{comm}$ if possible

\subsubsection{Vector Addition}

TODO

\subsubsection{Matrix Multiplication}

TODO

\subsubsection{Matrix-Vector Multiplication}

TODO

\subsection{Solving Systems of Equations}

\subsubsection{What are Systems of Equations?}

TODO

\subsubsection{Gaussian Elimination}

TODO: direct method

\subsubsection{Jacobi Iteration}

TODO: advantages of iteration methods over direct method

TODO: formulae and time complexity

TODO: serial and parallel

\subsubsection{Gauss-Seidel Iteration}

TODO: serial and parallel

\subsubsection{Lexicographic Gauss-Seidel}

TODO

\subsubsection{Red-Black Gauss-Seidel}

TODO

\subsubsection{Successive Over-Relaxation}

TODO: serial and parallel (same as Gauss-Seidel)

\subsubsection{Iteration Stopping Criteria}

TODO: error measures (3) and tolerance

TODO: matrix conditional number (residual)

\subsubsection{Improving Convergence Rates}

TODO: page 27 on lecture 7

\subsection{Partial Differential Equations}

\subsubsection{Derivatives and Approximating Derivatives}

TODO: with errors and stuff  due to floating point inaccuracy (dt too small). this means finding optimal value for dt

TODO: higher derivates

\subsubsection{Ordinary Differential Equations}

TODO

\subsubsection{Partial Derivatives}

TODO: what they are, maths definition and examples of application

\subsubsection{Discretising Partial Differential Equations}

TODO: refer to worksheet on this

\subsubsection{Initial Value Problems}

TIME	

TODO: ODE

TODO: examples of these problems, approximating them and using Euler's method

TODO: stiffness

\subsubsection{Boundary Value Problems}

SPATIAL

TODO: definition, examples of problems, approximating them using systems of equations

TODO: error measurement

TODO: boundary conditions

TODO: refer to boundary value problems worksheet

TODO: final notes from slides on it

\subsubsection{Heat Diffusion}

TODO: describe this steady state spatial problem

TODO: Laplace and Poision equations (1d)

TODO: Dirichlet boundary condition

TODO: discretising the partial differential equations (finite differences)

TODO: 2-space dimension and 2D poisson equation, with mesh present

TODO: discretising the problem and getting a tri-diagonal matrix

TODO: how diffusion smooyhs things out (due to averaging at each iteration)

TODO: how to partition, how the processes and each individual node on the mesh communicate with each other, speedup, efficiency and iso-efficiency of problem of heat diffusion for both row, block and red-black partitioning

\subsubsection{Initial-Boundary Value Problems}

TODO:  these are not just steady-state ptoblems! These are dynamic models which have variations in time as well as space

TODO: examples of these problems

TODO: time-dependant heat diffusion/averaging equation

TODO: advection

TODO: diffusion

TODO: reaction

TODO: diagrams showing behaviour of the different stuff

TODO: method of lines  (see paper notes on lecture 18 for this as well). This means using finite differences for spatial derivatives and explicit time-stepping with the forward Euler scheme.

TODO: using upwinding for space (not central differencing) and implicit time-stepping with the backward Euler method

TODO: applied example from slides

TODO: Improving Implicit Methods for Time

TODO: handling Multiple Space Dimensions

\subsection{Sparse Systems}

\subsubsection{1D Sparsity and Thomas Algorithm}

TODO: 1D pattern and reducing storage

TODO: Thomas algorithm for efficient Gaussian Elimination for a tridiagonal matrix

\subsubsection{2D Sparsity and banded Matrices}

TODO: central differencing

\subsubsection{Unstructured Meshes}

TODO: sparsity with unstructured meshes

\subsubsection{Storing Sparse Matrices}

TODO

\subsubsection{Bandwidth}

TODO

\subsubsection{Cuthill-McKee Ordering}

TODO: describe algorithm

TODO: give steps

TODO; give example

TODO: mention fill-in

\section{Meshes}

TODO: what a mesh is

\subsection{Structured Problems}

TODO: discuss structure of grids

TODO: definite regular/uniform, structured and unstructured meshes

\subsection{Unstructured Problems}

TODO: Talk about how you can have unstructered meshes

TODO: some example screenshots

TODO: discuss how you can triangulate any mesh and how it's often done

TODO: finally, say how WITH UNSTRUCTURED DATA YOU CAN MO LONGER HAVE A GRID AND MULTI-DIMENSIONAL RAYS CANNOT BE USED

\subsection{Mesh Representation}

TODO: what the names of the stuff are 

TODO: representation of unstructured meshes

\subsection{Process of Solving Unstructered Meshes}

TODO: list of how you load meshes in and all that other stuff and eventually solve

TODO: mention some of the algorithm strategies and how you load/represent the mesh in memory

(MAKE SURE TO GET LIST  FROM THE LAST PART

\subsection{Mesh Generation}

\subsubsection{Mesh Quality}

TODO: how there are many different ways to break up an unstructured mesh and a quality measure is required

TODO: mesh quality measures (two from slides)

\subsubsection{Edge Swappping}

TODO

\subsubsection{Voronoi Tesselation}

TODO: what it is

\subsubsection{Dual Graphs and Delaunay Triangulation}

TODO: dual graphs

TODO: Delaunay triangulation

\subsubsection{Point Insertion}

TODO

\subsubsection{Advancing Front Method}

TODO

\subsubsection{Algorithm Summary}

Unstructured Mesh Generation Algorithm Summary

TODO: grid-based approach on last slide

TODO: table with list of algorithms

\subsection{Mesh Partitioning}

\subsubsection{Good Partition Properties}

TODO: state how unstructured mesh is considered graph, partitions are subgraphs

\subsection{Dual Graphs}

TODO

\subsubsection{Recursive Coordinate Bisection (RBC)}

TODO: examples with and without edge weights

\subsubsection{Recursive Graph Bisection (RGB)}

TODO: examples with and without edge weights

\subsubsection{Multilevel Partitioning}

TODO: graph coarsening

\subsubsection{Other Algorithms}

TODO

\subsection{Adaptive Meshes}

TODO: discuss mesh resolution

\subsubsection{Uniform Refinement}

TODO

\subsubsection{Local Mesh Refinement}

TODO

TODO: discuss issues with these

\subsubsection{Local Error Estimation}

TODO: with uniform meshes

TODO: mesh patches

\subsubsection{Refining Triangular Meshes}

TODO: refining a single triangle

TODO: overall process of refining a triangular mesh

TODO: methods of refinement

\subsubsection{Dynamic Load Balancing and Difussion Method}

TODO: discuss general problems

TODO: DIFFUSION METHOD description, steps and examples`1

\section{Parallel Computing}

\subsection{Parallel Performance Theory}

\subsection{Definitions}

TODO: variable names and stuff

TODO: execution times (t\_comm, t\_comp, t\_startup, etc. what they all are)

\subsubsection{Speedup}

TODO: describe

TODO: Amdahl's law

TODO: Gustafson's law

TODO: state example problem and apply speedup

\subsubsection{Efficiency}

TODO: use same example problem as speedup

\subsubsection{Divide and Conquer}

TODO: something about this?

\subsubsection{Scalability and Iso-efficiency}

(on lecture 8)

TODO: use same example problem as speedup

TODO: use another example (Jacobi?)

\subsubsection{Time Complexity Analysis for Parallel Programs}

TODO: summary of how to analyse time complexity of parallel programs with reference to the WORKSHEET (step-by-step)

\subsection{Parallelising Sequential Operations}

TODO: pipelining

\subsection{Partitioning Strategies}

\subsubsection{Row/Strip Decomposition}

TODO: applications to major problems like heat diffusion and solving uniform linear systems

\subsubsection{Block Decomposition}

TODO

\section{MPI}

\subsection{Overview of MPI Paradigm}

TODO

\subsection{Sending and Receiving Messages}

TODO

\subsection{Synchronous and Asynchronous}

TODO

\subsection{Global Communication}

TODO: stuff like broadcast and map/reduce

\subsection{Loading Balancing}

TODO

\subsection{Example Parallel Problems}

TODO

\subsection{Building and Running MPI Programs}

TODO: MPD ring and machine specification

% Generate the bibliography.
\bibliography{latex-sample}
\bibliographystyle{unsrt}

\end{document}
